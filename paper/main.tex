\documentclass[11pt]{article}

% -------------------- packages --------------------
\usepackage[a4paper,margin=25mm]{geometry}
\usepackage[T1]{fontenc}
\usepackage[utf8]{inputenc}
\usepackage{lmodern}
\usepackage{microtype}

\usepackage{graphicx}
\usepackage{amsmath,amssymb}
\usepackage{booktabs}
\usepackage{caption}
\usepackage{subcaption}
\usepackage{float}
\usepackage{setspace}

\usepackage[numbers,sort&compress]{natbib}

\usepackage{hyperref}
\hypersetup{
  colorlinks=true,
  linkcolor=blue,
  citecolor=blue,
  urlcolor=blue
}

\onehalfspacing

% -------------------- title --------------------
\title{%
Regime Mapping of MEP-like Dissipation--Persistence Relations\\
in an Information-Using Artificial Life ABM with Information Cost
}
\author{Akihiko Itaya\\Independent Researcher}
\date{}

\begin{document}
\maketitle

% -------------------- abstract --------------------
\begin{abstract}
We study when a Maximal Entropy Production (MEP)-like association between dissipation and persistence advantage
appears or fails in an agent-based model (ABM) with explicit information-processing cost.
Agents adopt either a random policy or an informed policy that senses local gradients under noise and pays
an energetic information cost. We quantify persistence advantage by the paired difference in restricted mean
survival time ($\Delta$RMST) and dissipation-related activity by a paired dissipation proxy difference
($\Delta\sigma$-proxy), both defined as informed minus random under matched stochastic conditions.
By scanning environmental inflow at fixed cost and testing robustness across cost and noise, we identify four
inflow regimes characterized by the joint sign structure of $(\Delta\sigma\text{-proxy}, \Delta\text{RMST})$:
(i) a low-inflow trivial regime ($\Delta\sigma$-proxy$<0$, $\Delta$RMST$<0$),
(ii) a regime where the MEP-like association fails: dissipation proxy increases without conferring persistence advantage
($\Delta\sigma$-proxy$>0$, $\Delta$RMST$<0$),
(iii) a bounded regime where dissipation proxy and persistence advantage align
($\Delta\sigma$-proxy$>0$, $\Delta$RMST$>0$), and
(iv) a high-inflow regime where the MEP-like association fails in the opposite direction: persistence advantage persists while dissipation proxy decreases
($\Delta\sigma$-proxy$<0$, $\Delta$RMST$>0$).
To address concerns about proxy arbitrariness, we additionally analyze an independent alternative energetic proxy,
agent expenditure ($\Delta$AE), which excludes the field diffusion term and does not apply $T_0$ normalization;
we show that key low-inflow regime boundaries are preserved under $\Delta$AE.
Rather than providing a single counterexample, this study identifies both MEP-like and non-MEP-like regimes within the
same model by continuous parameter scanning, yielding an explicit parameter map of where an MEP-like association is observed
and where it fails in an information-using feedback system. Overall, increased dissipation is neither necessary nor sufficient
for persistence advantage in this class of ABMs. Here ``MEP-like'' is used strictly as a phenomenological label for regime-level correlation,
not as a thermodynamic maximizing principle.
\end{abstract}

% -------------------- introduction --------------------
\section{Introduction}
Links between entropy production, information use, and adaptation are discussed across nonequilibrium thermodynamics,
complex systems, and artificial life. A recurring hypothesis is that successful strategies tend to generate higher dissipation,
suggesting an MEP-like connection between energetic throughput and adaptive success. However, information-using agents typically
implement feedback: sensing affects action, action changes experienced environments, and the value of information depends on
environmental gradients. Under such feedback, increased energetic activity can arise without producing persistence advantage, and
conversely persistence can improve via reduced action when information becomes uninformative.

This paper provides a minimal ABM demonstration that the validity of an MEP-like association is parameter dependent. Rather than
treating MEP as universally true or false, we explicitly map the inflow ranges where a dissipation-related observable and persistence advantage align and
where they decouple. We do not test MEP as a maximizing principle; we map when an MEP-like correlation between a dissipation-related observable and persistence advantage emerges or fails under information-using feedback.

This work contributes to complex-systems methodology by providing a regime map of when a dissipation--persistence correlation emerges or fails in an information-using feedback system under systematic parameter scanning.


% -------------------- model --------------------
\section{Model}
Agents inhabit a 2D resource field on an $N\times N$ grid that receives a constant inflow at the center cell, undergoes diffusion,
and may lose resource through boundary handling. Each agent has a position $(x,y)$, velocity $(v_x,v_y)$, and internal energy $E$.
Agents consume local resource to increase $E$ and pay (i) a baseline maintenance cost and (ii) an additional information-processing cost
if they use an informed policy. Agents die when $E\le 0$ and divide when $E$ exceeds a threshold, paying a division cost.

The two policies are:
\begin{itemize}
  \item \textbf{Random:} stochastic drift without directional sensing.
  \item \textbf{Informed:} estimates local gradients under sensing noise and accelerates along the perceived gradient, paying an information cost.
\end{itemize}
We compare matched populations (random vs informed) under the same environmental inflow and stochastic seeds.

% -------------------- measures --------------------
\section{Measures}

\paragraph{Persistence advantage (paired $\Delta$RMST).}
Persistence advantage is quantified by the paired difference in restricted mean survival time:
\[
\Delta \mathrm{RMST}(I) \;=\; \mathrm{RMST}_{\text{informed}}(I) \;-\; \mathrm{RMST}_{\text{random}}(I).
\]
Positive $\Delta$RMST indicates that informed agents persist longer on average within the observation window.
We do not claim $\Delta$RMST constitutes evolutionary fitness in a genetic sense; rather, it is an ecological persistence metric
that isolates survival consequences of fixed strategies under matched environments. Throughout, ``advantage'' refers to ecological
persistence under fixed strategies rather than evolutionary selection with heredity.

\paragraph{Operational dissipation proxy (paired $\Delta\sigma$-proxy).}
We use an operational dissipation-related proxy capturing irreversible energetic throughput in the open agent--environment system as
explicitly modeled in the simulator. We emphasize that this proxy is \emph{not} a thermodynamic entropy production rate; it is an
operational observable assembled from model-defined nonnegative diffusion losses and explicit energetic expenditures at the agent interface.

Let the resource field be $u_i(t)\ge 0$ on grid cells $i\in\Omega$ and agent internal energies be $E_a(t)$. Define a temperature-scale
constant
\[
T_0=\max(10^{-9},T_{\rm bath}),
\]
where $T_{\rm bath}$ is the simulator parameter (\texttt{bathTemp}). The dissipation proxy is defined per time step by
\[
\sigma\text{-proxy}(t)=\sigma_{\rm diff}(t)+\sigma_{\rm act}(t).
\]

\textbf{(i) Diffusion proxy (nonnegative by construction).}
With diffusion coefficient $\kappa$ (\texttt{diffusionKappa}) and time step $dt$, for each nearest-neighbor edge $\langle i,j\rangle$
(on a 4-neighborhood) define $\Delta u_{ij}(t)=u_i(t)-u_j(t)$. The simulator accumulates
\[
\sigma_{\rm diff}(t)=\sum_{\langle i,j\rangle}\frac{\kappa\,(\Delta u_{ij}(t))^2\,dt}{T_0},
\]
which is nonnegative because it is a sum of squared differences.

\textbf{(ii) Agent expenditure term.}
The simulator defines the total explicit energetic expenditure at agents as
\[
Q_{\rm act}(t)=Q_{\rm maint}(t)+Q_{\rm info}(t)+Q_{\rm div}(t),
\]
and sets
\[
\sigma_{\rm act}(t)=\frac{Q_{\rm act}(t)}{T_0}.
\]
Therefore,
\[
\sigma\text{-proxy}(t)=\sigma_{\rm diff}(t)+\frac{Q_{\rm act}(t)}{T_0}.
\]
We report the paired difference
\[
\Delta \sigma\text{-proxy}(I)=\sigma\text{-proxy}_{\text{informed}}(I)-\sigma\text{-proxy}_{\text{random}}(I).
\]
Our claims concern the regime-level \emph{sign structure} of dissipation--persistence relations, not the precise physical value of entropy production.

\paragraph{Alternative energetic proxy: agent expenditure (AE).}
To address concerns about arbitrariness of $\sigma$-proxy, we also analyze an independent alternative proxy that excludes the diffusion term
and does not apply $T_0$ normalization:
\[
\mathrm{AE}(t)=Q_{\rm maint}(t)+Q_{\rm info}(t)+Q_{\rm div}(t)=Q_{\rm act}(t),
\]
and we report
\[
\Delta \mathrm{AE}(I)=\mathrm{AE}_{\text{informed}}(I)-\mathrm{AE}_{\text{random}}(I).
\]
Thus, $\mathrm{AE}$ corresponds exactly to the agent-side expenditure underlying $\sigma_{\rm act}$, but without diffusion and without $T_0$ scaling.
Agreement of regime boundaries under $\Delta\sigma$-proxy and $\Delta$AE indicates that the low-inflow sign-flip is not an artifact of the diffusion
term or normalization.

\paragraph{Explicit cost components (as implemented).}
Each alive agent pays a baseline maintenance cost per step:
\[
q_{\rm maint}^{(a)}(t)=m_0\,dt,\qquad m_0=0.2,
\]
so $Q_{\rm maint}(t)=N(t)m_0dt$ with $N(t)$ the number of alive agents.

For informed agents, four-point sensing is performed at a look distance $L$ (\texttt{lookDist}) with noise:
\[
\tilde u = u+\eta\xi,\qquad \xi\sim\mathcal N(0,1),\ \eta=\texttt{sensingNoise},
\]
and gradient estimates use the sensed differences
\[
g_x=\tilde u_{\rm right}-\tilde u_{\rm left},\qquad g_y=\tilde u_{\rm down}-\tilde u_{\rm up}.
\]
The informed velocity update is $v\leftarrow v+\alpha(g_x,g_y)dt$ with $\alpha=0.5$ in the implementation.
The information-processing cost per informed agent is
\[
q_{\rm info}^{(a)}(t)=\Bigl(\lambda(|g_x|+|g_y|)+0.2\lambda\Bigr)dt,\qquad \lambda=\texttt{intelligenceCost}.
\]
Division events pay a fixed division cost $c_{\rm div}$ (\texttt{divisionCost}) whenever $E\ge E_{\rm div}$ (\texttt{divisionThreshold}).

\paragraph{Uncertainty.}
All curves are shown with bootstrap confidence intervals (95\% CI).

% -------------------- experimental design & inference --------------------
\section{Experimental design and inference}

\paragraph{Matched stochastic conditions (paired design).}
For each inflow value $I$, we run both policies (random and informed) under matched stochastic conditions by using identical seeds for the simulator RNG.
We use seedbase $=1000$ and $S=50$ seeds: $s\in\{1000,1001,\dots,1049\}$.
Each seed defines one paired observation of $(\sigma\text{-proxy},\mathrm{AE},\mathrm{RMST})$ for informed and random, and we analyze paired differences
$\Delta\sigma$-proxy, $\Delta$AE, and $\Delta$RMST.

\paragraph{RMST and bootstrap CI.}
RMST is computed over a fixed observation window (simulation horizon) using standard restricted mean survival time estimation.
Confidence intervals for $\Delta$RMST and for energetic differences ($\Delta\sigma$-proxy, $\Delta$AE) are computed by nonparametric bootstrap over the $S=50$
paired seed-level differences (percentile CI unless otherwise stated in figure generation scripts).

\paragraph{Boundary (zero-crossing) estimation.}
Regime boundaries are operationally defined as inflow values where the sign of a paired difference changes.
For visualization, we report point estimates of zero-crossings by linear interpolation between adjacent inflow grid points where the mean curve changes sign:
\[
\hat I_0= I_k + (I_{k+1}-I_k)\frac{-\bar y(I_k)}{\bar y(I_{k+1})-\bar y(I_k)}.
\]
Primary inference relies on the confidence bands and (when used) bootstrap sign probabilities; boundary locations should be interpreted as statistical transition
points rather than sharp physical thresholds. In the main text we report rounded boundary values (e.g., $\approx 0.01$, $\approx 0.094$) and provide
more precise point estimates in close-up figures and/or appendices.

% -------------------- regime definitions --------------------
\section{Inflow regimes and MEP-validity map}
We define four inflow regimes using the joint sign structure of $(\Delta \sigma\text{-proxy}, \Delta\mathrm{RMST})$
and interpret them through an \emph{MEP-validity} lens. Throughout, ``MEP-like'' refers to the empirical alignment
``higher dissipation proxy accompanies higher persistence advantage'' (both positive), not to a universal physical law.

\begin{itemize}
  \item \textbf{R0: Trivial low-inflow regime} \\
  \textbf{Signature:} $\Delta \sigma\text{-proxy} < 0$ and $\Delta\mathrm{RMST} < 0$ (inflow $\approx 0$--$0.01$). \\
  \textbf{Interpretation:} Informed agents are disfavored and also dissipate less. This is a degenerate limit where gradients are
  insufficient for functional information use; we treat it as a trivial baseline rather than evidence \emph{for} MEP.

  \item \textbf{R1: failure of the MEP-like association regime I (dissipation without advantage)} \\
  \textbf{Signature:} $\Delta \sigma\text{-proxy} > 0$ while $\Delta\mathrm{RMST} < 0$ (inflow $\approx 0.01$--$0.095$). \\
  \textbf{Interpretation:} Dissipation proxy increases under informed behavior without yielding persistence advantage.

  \item \textbf{R2: Bounded MEP-like regime (conditional validity)} \\
  \textbf{Signature:} $\Delta \sigma\text{-proxy} > 0$ and $\Delta\mathrm{RMST} > 0$ (inflow $\approx 0.095$--$0.16$). \\
  \textbf{Interpretation:} Dissipation proxy and persistence advantage align, producing a bounded MEP-like window.

  \item \textbf{R3: failure of the MEP-like association regime II (low-dissipation advantage under abundance)} \\
  \textbf{Signature:} $\Delta \sigma\text{-proxy} < 0$ while $\Delta\mathrm{RMST} > 0$ (inflow $\gtrsim 0.16$). \\
  \textbf{Interpretation:} Informed agents retain a persistence advantage under fixed strategies while dissipating less than random agents.
\end{itemize}

We emphasize that these regime boundaries are operational and inferred from sign changes in the inflow scans and close-ups shown below.
To test robustness against proxy arbitrariness, we additionally examine $\Delta$AE and show that key low-inflow transitions are preserved under this
alternative, diffusion-free, non-normalized energetic proxy.

% -------------------- results --------------------
\section{Results}

\subsection{Global inflow scan and regime structure (fixed cost)}
Figure~\ref{fig:global} shows the global inflow scan at fixed cost ($c=0.2$), exhibiting the full sequence R0--R3 when interpreted by sign structure.
The figure makes explicit that an MEP-like association emerges only in a bounded inflow window (R2), and fails again outside that window.

\begin{figure}[H]
\centering
\includegraphics[width=0.95\linewidth]{figures/fig1.pdf}
\caption{
\textbf{Global inflow scan and MEP-validity map (fixed cost $c=0.2$).}
Paired differences $\Delta \sigma$-proxy and $\Delta$RMST as functions of inflow.
Interpreted as an explicit parameter map: R0 (trivial low-inflow baseline), R1 (failure of the MEP-like association: $\Delta\sigma$-proxy$>0$ but $\Delta$RMST$<0$),
R2 (bounded MEP-like window: both positive), and R3 (failure of the MEP-like association under abundance: $\Delta\sigma$-proxy$<0$ while $\Delta$RMST$>0$).
}
\label{fig:global}
\end{figure}

\subsection{Robustness against proxy arbitrariness: $\Delta$AE as an alternative energetic proxy}
To address concerns that the diffusion-based term or the $T_0$ normalization could drive the observed low-inflow sign changes in $\Delta\sigma$-proxy,
we evaluate an independent energetic proxy, $\Delta$AE, which excludes diffusion and does not normalize by $T_0$.

Figure~\ref{fig:AE_global} shows the global behavior of $\Delta$AE (paired informed minus random). Figure~\ref{fig:AE_overlay} provides a compact overlay
of $\Delta$AE (scaled by a factor 20 for visibility) and $\Delta$RMST. Scaling is purely for visualization and does not affect sign or boundary inference.

\begin{figure}[H]
\centering
\includegraphics[width=0.95\linewidth]{figures/fig2.pdf}
\caption{
\textbf{Alternative proxy: global $\Delta$AE vs inflow.}
$\Delta$AE is the paired difference in agent expenditure (maintenance + information + division) between informed and random strategies.
Unlike $\Delta\sigma$-proxy, $\Delta$AE excludes the field diffusion term and does not apply $T_0$ normalization.
The preservation of low-inflow transition structure under $\Delta$AE supports robustness against proxy arbitrariness.
}
\label{fig:AE_global}
\end{figure}

\begin{figure}[H]
\centering
\includegraphics[width=0.95\linewidth]{figures/fig3.pdf}
\caption{
\textbf{Overlay of $\Delta$AE and $\Delta$RMST (visual scaling noted).}
$\Delta$AE (orange) is multiplied by 20 solely for visibility; $\Delta$RMST is shown in blue with corresponding 95\% CIs.
The $y=0$ line is shown for reference. Scaling does not affect sign-based regime identification.
}
\label{fig:AE_overlay}
\end{figure}

\subsection{Boundary A: R0 $\rightarrow$ R1 (sign flip of $\Delta\sigma$-proxy near inflow $\sim 0.01$)}
Figure~\ref{fig:R0R1} resolves the low-inflow transition where $\Delta \sigma$-proxy crosses from negative to positive.
In this neighborhood, $\Delta$RMST remains negative, marking entry into the first failure-of-association regime (R1).

\begin{figure}[H]
\centering
\begin{subfigure}{0.48\linewidth}
\includegraphics[width=\linewidth]{figures/fig4a.pdf}
\caption{$\Delta \sigma$-proxy (95\% CI)}
\end{subfigure}
\hfill
\begin{subfigure}{0.48\linewidth}
\includegraphics[width=\linewidth]{figures/fig4b.pdf}
\caption{$\Delta$RMST (95\% CI)}
\end{subfigure}
\caption{
\textbf{Boundary A: R0 $\rightarrow$ R1.}
Near inflow $\sim 0.01$, $\Delta \sigma$-proxy crosses zero (negative to positive) while $\Delta$RMST remains negative,
transitioning from R0 to R1 where dissipation proxy increases without conferring persistence advantage.
Boundary locations are operationally estimated by the zero-crossing of the mean curve (linear interpolation) and supported by the 95\% CI bands.
}
\label{fig:R0R1}
\end{figure}

\subsection{Boundary A under an independent proxy: $\Delta$AE close-up}
Figure~\ref{fig:AE_R0R1} shows the corresponding low-inflow neighborhood for $\Delta$AE.
The preservation of the sign flip under $\Delta$AE indicates that the R0$\rightarrow$R1 transition is not an artifact of the diffusion proxy term or $T_0$ normalization.

\begin{figure}[H]
\centering
\includegraphics[width=0.85\linewidth]{figures/fig5.pdf}
\caption{
\textbf{Boundary A under an alternative energetic proxy ($\Delta$AE).}
Close-up view of $\Delta$AE around the low-inflow transition.
$\Delta$AE excludes diffusion and does not normalize by $T_0$, yet exhibits a consistent low-inflow sign flip.
This supports robustness of the R0$\rightarrow$R1 boundary against proxy arbitrariness.
}
\label{fig:AE_R0R1}
\end{figure}

\subsection{Boundary B: R1 $\rightarrow$ R2 (onset of survival advantage near inflow $\sim 0.095$)}
Figure~\ref{fig:R1R2} shows the transition where $\Delta$RMST crosses from negative to positive while $\Delta \sigma$-proxy stays positive.
This boundary defines onset of the bounded MEP-like window (R2).

\begin{figure}[H]
\centering
\begin{subfigure}{0.48\linewidth}
\includegraphics[width=\linewidth]{figures/fig6a.pdf}
\caption{$\Delta \sigma$-proxy (95\% CI)}
\end{subfigure}
\hfill
\begin{subfigure}{0.48\linewidth}
\includegraphics[width=\linewidth]{figures/fig6b.pdf}
\caption{$\Delta$RMST (95\% CI)}
\end{subfigure}
\caption{
\textbf{Boundary B: R1 $\rightarrow$ R2.}
Near inflow $\sim 0.095$, $\Delta$RMST crosses zero while $\Delta \sigma$-proxy remains positive, marking onset of a bounded regime (R2)
where dissipation proxy and persistence advantage align (MEP-like window).
}
\label{fig:R1R2}
\end{figure}

\subsection{Boundary C: R2 $\rightarrow$ R3 (sign flip of $\Delta\sigma$-proxy near inflow $\sim 0.16$)}
Figure~\ref{fig:R2R3} shows the high-inflow transition where $\Delta \sigma$-proxy crosses from positive to negative while $\Delta$RMST remains positive.
This defines the abundant-inflow regime (R3), where informed agents retain a persistence advantage under fixed strategies despite exhibiting a lower dissipation proxy than random agents.

\begin{figure}[H]
\centering
\begin{subfigure}{0.48\linewidth}
\includegraphics[width=\linewidth]{figures/fig7a.pdf}
\caption{$\Delta \sigma$-proxy (95\% CI)}
\end{subfigure}
\hfill
\begin{subfigure}{0.48\linewidth}
\includegraphics[width=\linewidth]{figures/fig7b.pdf}
\caption{$\Delta$RMST (95\% CI)}
\end{subfigure}
\caption{
\textbf{Boundary C: R2 $\rightarrow$ R3.}
At higher inflow (near $\sim 0.16$), $\Delta \sigma$-proxy transitions from positive to negative while $\Delta$RMST remains positive,
defining R3: an abundant-inflow regime where informed agents retain a persistence advantage under fixed strategies while dissipating less than random agents.
}
\label{fig:R2R3}
\end{figure}

\subsection{Robustness across information cost}
Figure~\ref{fig:cost} shows that the inflow-dependent sign/regime structure persists over a wide range of information costs.

\begin{figure}[H]
\centering
\begin{subfigure}{0.48\linewidth}
\includegraphics[width=\linewidth]{figures/fig8a.pdf}
\caption{$\Delta \sigma$-proxy vs inflow (by cost)}
\end{subfigure}
\hfill
\begin{subfigure}{0.48\linewidth}
\includegraphics[width=\linewidth]{figures/fig8b.pdf}
\caption{$\Delta$RMST vs inflow (by cost)}
\end{subfigure}
\caption{
\textbf{Robustness to information-processing cost.}
Overlay of inflow sweeps across costs shows that the qualitative regime ordering (R0--R3) is preserved. Costs primarily shift amplitudes and
transition locations without eliminating the bounded MEP-like window and the two regimes of failure of the MEP-like association.
}
\label{fig:cost}
\end{figure}

\subsection{Robustness across stochastic noise}
Figure~\ref{fig:noise} shows that increased noise attenuates signal magnitude but preserves the overall sign/regime ordering.

\begin{figure}[H]
\centering
\begin{subfigure}{0.48\linewidth}
\includegraphics[width=\linewidth]{figures/fig9a.pdf}
\caption{$\Delta \sigma$-proxy vs inflow (by noise)}
\end{subfigure}
\hfill
\begin{subfigure}{0.48\linewidth}
\includegraphics[width=\linewidth]{figures/fig9b.pdf}
\caption{$\Delta$RMST vs inflow (by noise)}
\end{subfigure}
\caption{
\textbf{Robustness to noise (fixed cost $c=0.2$).}
As noise increases, both $\Delta \sigma$-proxy and $\Delta$RMST signals weaken and uncertainty grows, but the qualitative inflow-dependent regime
structure remains identifiable.
}
\label{fig:noise}
\end{figure}

% -------------------- discussion --------------------
\section{Discussion}

\subsection{What is new: an explicit parameter map of MEP-like validity and failure}
The central contribution of this work is an explicit parameter-dependent map showing where an MEP-like
association appears and where it fails within the same ABM. This reframes the MEP debate from a universal yes/no
question to a regime-dependent ecological question.
While regime identification is based on sign structure, the accompanying confidence bands indicate that effect magnitudes vary continuously with inflow,
highlighting that regime transitions are gradual rather than sharp phase transitions.

\subsection{Proxy meaning, transparency, and robustness}
We explicitly define $\sigma$-proxy as a sum of (i) a nonnegative diffusion proxy proportional to squared local resource differences and
(ii) explicit agent expenditures normalized by a temperature scale $T_0$. This definition is fully reproducible from the simulator implementation.
Importantly, we do not claim $\sigma$-proxy is a thermodynamic entropy production rate. Instead, it is an operational dissipation-related observable that
aggregates model-defined irreversible throughput at the agent--environment interface.

To defend against concerns that the observed low-inflow transition could be an artifact of proxy design (e.g., inclusion of diffusion or $T_0$ normalization),
we introduced an independent alternative proxy, agent expenditure (AE), which excludes diffusion and does not normalize by $T_0$.
The preservation of the low-inflow sign flip under $\Delta$AE provides direct evidence that the R0$\rightarrow$R1 boundary is not driven solely by
a particular decomposition or scaling choice in $\sigma$-proxy.

\subsection{``Maximum'' versus correlation}
Our analysis concerns correlation structure, not maximization among steady states. ``MEP-like'' here denotes a bounded inflow window where higher dissipation-related
proxy co-occurs with higher persistence advantage. We interpret MEP-like behavior as an emergent correlation produced by information--environment coupling,
not as an optimizing principle of the underlying dynamics.

\subsection{Mechanistic interpretation and limits}
The two regimes of failure of the MEP-like association are naturally explained by information-dependent feedback:
dissipation without benefit when gradients are weakly informative, and reduced dissipation advantage when gradients vanish under abundance.
A limitation is that we do not directly quantify information relevance (e.g., gradient-following accuracy or SNR) in the present figures; adding such
mechanism-resolving observables is an important next step and would further strengthen causal interpretation.

\subsection{Implications for Artificial Life and extensions}
Information relevance, not energetic maximization per se, structures persistence advantages in this class of ABMs.
We focus on fixed-strategy comparisons to isolate informational feedback effects; extending the regime map to evolving populations (mutation/selection)
is a natural next step. Likewise, exploring alternative field geometries (asymmetry, multiple peaks, time-varying inflow) would clarify which aspects of the
regime structure are generic versus geometry dependent.

\subsection{Reproducibility}
To support reproducibility, we will release the simulation code, parameter settings, and the exact outputs used for all figures (including raw and aggregated tables,
survival tables, and run metadata with seeds) in a versioned public repository with a pinned commit hash upon publication.

\subsection{AI-assisted modeling and epistemic responsibility}
The present study was conducted in close interaction with large language models (LLMs), which were used to assist in the construction of the simulation code,
the formulation of experimental designs, and the drafting of the manuscript. The initial motivation and guiding hypothesis---namely, that evolutionary success in open systems
may be associated with increased rates of entropy production---were provided by the author.

Throughout the project, LLMs were employed as generative tools to explore possible modeling choices and analytical formulations. The author evaluated these suggestions,
selected among alternatives, and interpreted the resulting behaviors in light of the research questions. All responsibility for the modeling assumptions, experimental settings,
interpretation of results, and scientific claims rests with the author. The use of AI tools in this work is disclosed to contribute to ongoing discussions about authorship,
agency, and epistemic responsibility in computational and artificial life research.

% -------------------- conclusion --------------------
\section{Conclusion}
We analyzed an ABM comparing random and informed strategies under explicit information-processing cost. By scanning environmental inflow, we identified four
inflow regimes characterized by the sign structure of $(\Delta\sigma\text{-proxy}, \Delta\mathrm{RMST})$.
An MEP-like alignment (both positive) appears only in a bounded inflow window, while dissipation-related activity and persistence advantage decouple in two distinct
failure-of-association regimes. We further introduced an independent alternative energetic proxy, $\Delta$AE, excluding diffusion and $T_0$ normalization, and showed that
key low-inflow transition structure persists, supporting robustness against proxy arbitrariness.
These findings indicate that apparent MEP behavior in information-using adaptive systems is best interpreted as a contingent, context-dependent correlation rather than a universal organizing principle.

% -------------------- references --------------------
\bibliographystyle{plainnat}
\begin{thebibliography}{9}

\bibitem{landauer1961irreversibility}
R.~Landauer.
\newblock Irreversibility and heat generation in the computing process.
\newblock \emph{IBM Journal of Research and Development}, 5:183--191, 1961.

\bibitem{parrondo2015thermodynamics}
J.~M.~R. Parrondo, J.~M. Horowitz, and T.~Sagawa.
\newblock Thermodynamics of information.
\newblock \emph{Nature Physics}, 11:131--139, 2015.

\bibitem{dewar2003information}
R.~C. Dewar.
\newblock Information theory explanation of the fluctuation theorem, maximum entropy production and self-organized criticality.
\newblock \emph{Journal of Physics A}, 36(3):631--641, 2003.

\bibitem{martyushev2006maxentprod}
L.~M. Martyushev and V.~D. Seleznev.
\newblock Maximum entropy production principle in physics, chemistry and biology.
\newblock \emph{Physics Reports}, 426(1):1--45, 2006.

\end{thebibliography}

\end{document}
